
% Updated version incorporating the two small corrections
\documentclass[11pt]{article}
\usepackage{amsmath,amssymb,amsthm,enumitem}
\usepackage[a4paper,margin=3cm]{geometry}
\usepackage{bm}

\usepackage[utf8]{inputenc}   % allow UTF-8 in the file
\usepackage{amsmath,amsfonts,amssymb}
\usepackage{enumitem}         % for (i), (ii) labels
\usepackage{geometry}         % nicer margins (optional)
\newtheorem{theorem}{Theorem}

\begin{document}


\title{Selected problems from Foundations of Modern Probability}
Here i am to show some solutions to selected problems from the book by Kallenberg.

\section{Sets and functions, measures and integration}
\subsection*{Problem 3}
For any space $S$ let $\mu A$ denote the cardinality of a set $A \subset S$. Show that $\mu$ is a measure in $(S, 2^S)$.
\begin{proof}
We apply the definition of a measure.
\textbf{1. Null empty set:}  
We have
\[
\mu(\emptyset) = \#\emptyset = 0.
\]
\textbf{2. Countable additivity:}  
Let \( (A_k)_{k \geq 1} \subset 2^S \) be a countable sequence of pairwise disjoint subsets of \( S \). Then their union is:
\[
\mu\left( \bigcup_{k \geq 1} A_k \right) = \#\left( \bigcup_{k \geq 1} A_k \right).
\]
Since the sets \( A_k \) are disjoint, each element in the union belongs to exactly one \( A_k \), so the cardinality of the union is the sum of the cardinalities:
\[
\#\left( \bigcup_{k \geq 1} A_k \right) = \sum_{k=1}^\infty \#A_k = \sum_{k=1}^\infty \mu(A_k).
\]

Hence, \( \mu \) is countably additive.  \\
\end{proof}


\subsection*{Problem 10 -- Fubini--Tonelli with counting measure}

Let $(S,\mathcal{S},\mu)$ be a $\sigma$--finite measure space and let
\[
\bigl(\mathbb{N},\,2^{\mathbb{N}},\,\nu\bigr), \qquad \nu(A)=\#A,
\]
be the counting--measure space on the natural numbers.
Write $\mu\otimes\nu$ for the product measure on
$\bigl(S\times\mathbb{N},\,\mathcal{S}\otimes 2^{\mathbb{N}}\bigr)$.

\medskip
\textbf{Theorem (Tonelli--Fubini).}\;
Let \(f:S\times\mathbb{N}\to[-\infty,\infty]\) be
\(\mathcal{S}\otimes 2^{\mathbb{N}}\)-measurable.

\begin{enumerate}[label=(\roman*)]
\item (\emph{Tonelli}) If \(f\ge 0\), then
\begin{equation}\label{eq:tonelli}
   \int_{S\times\mathbb{N}} f(s,t)\,(\mu\otimes\nu)(ds,dt)
   \;=\;
   \int_S \Bigl(\sum_{t\in\mathbb{N}} f(s,t)\Bigr)\,\mu(ds)
   \;=\;
   \sum_{t\in\mathbb{N}} \int_S f(s,t)\,\mu(ds),
\end{equation}
allowing the value \(+\infty\).

\item (\emph{Fubini}) If \(f\in L^{1}(\mu\otimes\nu)\),  
all three integrals in \eqref{eq:tonelli} are finite and equal, and both
iterated integrals exist as absolutely convergent expressions.
\end{enumerate}

\begin{proof}
\break
\textbf{Step 1. Indicator rectangles.}\;
Let \(A=B\times C\) with \(B\in\mathcal{S}\) and \(C\subseteq\mathbb{N}\).
For the indicator \(\mathbf{1}_{A}(s,t)=\mathbf{1}_{B}(s)\,\mathbf{1}_{C}(t)\) we have
\[
\int_{S\times\mathbb{N}} \mathbf{1}_{A}\,d(\mu\otimes\nu)
   =(\mu\otimes\nu)(A)=\mu(B)\,\nu(C).
\]
On the other hand,
\[
\int_S \sum_{t\in\mathbb{N}}\mathbf{1}_{A}(s,t)\,\mu(ds)
   =\int_S \mathbf{1}_{B}(s)\Bigl(\sum_{t\in C}1\Bigr)\mu(ds)
   =\mu(B)\,\#C
   =\mu(B)\,\nu(C),
\]
and the same equality holds if the order of integration and summation is
reversed.  Hence \eqref{eq:tonelli} holds for \(\mathbf{1}_{A}\).

\smallskip
\textbf{Step 2. Simple functions.}\;
Any non‑negative simple function can be written
\(f=\sum_{k=1}^{m}c_k\,\mathbf{1}_{A_k}\) with \(c_k\ge 0\) and \(A_k\) rectangles
as above.  By linearity of the integral and Step 1, \eqref{eq:tonelli}
holds for every such \(f\).

\smallskip
\textbf{Step 3. Non‑negative measurable functions.}\;
For a general \(f\ge 0\) choose an increasing sequence of simple
functions \((f_n)_{n\ge 1}\) with \(f_n\uparrow f\).
Applying the Monotone Convergence Theorem on both sides of
\eqref{eq:tonelli} and using Step 2 yields the Tonelli identity for \(f\).

\smallskip
\textbf{Step 4. Integrable functions.}\;
If \(f\in L^{1}(\mu\otimes\nu)\), decompose \(f=f^{+}-f^{-}\) with
\(f^{\pm}\ge 0\) and \(f^{\pm}\in L^{1}\).
Apply Step 3 to \(f^{+}\) and \(f^{-}\) separately and subtract;
finiteness follows from integrability.  Thus \eqref{eq:tonelli} holds
and all integrals are finite.

\smallskip
Steps 1--4 establish Tonelli’s theorem for \(f\ge 0\) and Fubini’s
theorem for \(f\in L^{1}(\mu\otimes\nu)\).
\end{proof}
















\end{document}

