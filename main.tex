
% Updated version incorporating the two small corrections
\documentclass[11pt]{article}
\usepackage{amsmath,amssymb,amsthm,enumitem}
\usepackage[a4paper,margin=3cm]{geometry}
\usepackage{bm}

\newtheorem{theorem}{Theorem}

\begin{document}

\begin{theorem}[Negative answer to Thorisson Problem~3.1]
Let $Y=(Y_k)_{k\in\mathbb Z}$ be an \emph{arbitrary} stationary $\{0,1\}$‑valued process.
Let $(p_k)_{k\ge0}\subset(0,1)$ satisfy
\begin{subequations}\label{eq:pconds}
\begin{align}
  p_k &\xrightarrow{k\to\infty} 0, \label{eq:p1}\\
  c  &:= \liminf_{n\to\infty}\sum_{k=n}^{2n}p_k \;>\;0. \label{eq:p2}
\end{align}
\end{subequations}
Let $(J_k)_{k\in\mathbb Z}$ be independent $\mathrm{Bernoulli}(p_k)$ variables, independent of $Y$, and set
\[
  X_k := Y_k \oplus J_k,\qquad
\]
where $\oplus$ denotes addition modulo~$2$ (the XOR–operator on $\{0,1\}$).
Denote by $\theta_n$ the time–shift: $(\theta_n\bm x)_t := x_{t+n}$.
Then
\begin{enumerate}[label=\textup{(\alph*)}]
  \item \textbf{Setwise convergence:}
        \(
          \displaystyle
          \lim_{n\to\infty} P(\theta_n X\in A)=P(Y\in A)
          \quad\forall A\in\mathcal E^{\infty}.
        \)
  \item \textbf{No total–variation convergence:}
        \(
          \displaystyle
          \liminf_{n\to\infty}
          \bigl\|\,P(\theta_n X)-P(Y)\bigr\|_{\mathrm{TV}}
          \;\ge\;1-e^{-c}\;>\;0.
        \)
\end{enumerate}
Hence setwise convergence of the shifted laws does not imply convergence in total variation.
\end{theorem}

\begin{proof} \\
\textit{Step 1 (setwise convergence).}
Fix a cylinder \(A\) that depends on the coordinates \(t_{1}<\dots<t_{m}\).
Because \(X_{k}\neq Y_{k}\) exactly when \(J_{k}=1\),
\[
  \bigl|P(\theta_{n}X\in A)-P(Y\in A)\bigr|
     \le\sum_{i=1}^{m}p_{\,n+t_{i}}
     \xrightarrow{n\to\infty}0\qquad(\text{by }\eqref{eq:p1}).
\]
Let \(\mathcal C\) denote the \emph{finite cylinder sets}; this family is a
$\pi$-system since it is closed under finite intersections.
Set
\[
  \mathcal D:=\bigl\{A\subset\{0,1\}^{\mathbb Z} :
	P(\theta_{n}X\in A) \xrightarrow{n\to\infty}P(Y\in A)\bigr\}.
\]
\(\mathcal D\) fulfills the properties of a Dynkin system, and is then one.
The estimate above shows \(\mathcal C\subset\mathcal D\).
Hence the $\pi - \lambda$ lemma gives
\[
  \sigma(\mathcal C)=\mathcal E^{\infty}\subset\mathcal D,
\]
so \(P(\theta_{n}X\in A)\to P(Y\in A)\) for every
\(A\in\mathcal E^{\infty}\); that is, setwise convergence holds. 
\newline
\textit{Step 2 (failure of total variation).}
For fixed \(n\in\mathbb N\) condition on the entire path \(Y=y\) and put  
\[
  \mu_y:=\mathcal L(\theta_n X\mid Y=y),\qquad
  \nu_y:=\delta_y .
\]
Only the coordinates \(n,\dots,2n\) can differ, so
\[
  \|\mu_y-\nu_y\|_{\mathrm{TV}}
      \;=\;
      P\!\bigl((J_n,\dots,J_{2n})\neq(0,\dots,0)\bigr)
      \;=\;
      1-\prod_{k=n}^{2n}(1-p_k)
      \;\ge\;
      1-\exp\!\Bigl(-\sum_{k=n}^{2n}p_k\Bigr).
\]
Using \eqref{eq:p2} and taking \(\liminf_{n\to\infty}\) we obtain
\[
  \liminf_{n\to\infty}
  \bigl\|P(\theta_n X)-P(Y)\bigr\|_{\mathrm{TV}}
     \;\ge\;1-e^{-c}\;>\;0,
\]
which proves~(b).
\end{proof}

\end{document}

